%!TEX root = ../thesis.tex
%*******************************************************************************
%*********************************** First Chapter *****************************
%*******************************************************************************

\chapter{Introduction}  %Title of the First Chapter

\ifpdf
    \graphicspath{{Chapter1/Figs/Raster/}{Chapter1/Figs/PDF/}{Chapter1/Figs/}}
\else
    \graphicspath{{Chapter1/Figs/Vector/}{Chapter1/Figs/}}
\fi

\label{chapter 1}

%****************************************************************************

\section{Powerful Nuclear Fusion}
\label{section1.1}
 Nuclear processes can release enormous amounts of energy, notably through two mechanisms: fission and fusion. Fission involves the splitting of heavy atomic nuclei and is most famously utilized in atomic bombs. In contrast, fusion, which entails the merging of light atomic nuclei—typically hydrogen and its isotopes into helium—releases significantly more energy than fission. The first uncontrolled demonstration of nuclear fusion was the detonation of hydrogen bombs. An illustrative example is the Tsar Bomba  \cite{TsarBomba}, which was detonated by the Soviet Union on Novaya Zemlya Island on October 30, 1961. Weighing approximately 27 tonnes, the Tsar Bomba's yield was intentionally reduced from its theoretical maximum to lessen environmental impacts and to safeguard the delivery crew. Despite this reduction, the bomb's explosion was still about 1570 times more powerful than that of the combined fission bombs used during World War II, underscoring the immense potential of nuclear fusion as an energy source. This demonstration highlights the scale of energy that nuclear fusion can unleash, which is pivotal to understanding the technology's promise for power generation, particularly in applications such as tokamaks.

\section{Controlled Fusion}
\label{section1.2}
Fusion energy is appealing due to its high power output, and the fuel it requires, hydrogen isotopes, is both abundant and poses minimal radioactive risk \cite{moynihan2023fusion}. However, to harness this energy effectively, controlled conditions are necessary for the fusion process. Currently, the main methods of controlled nuclear fusion being explored include Magnetic Confinement Fusion (MCF), Inertial Confinement Fusion (ICF), and Magnetized Target Fusion (MTF). During the fusion process, the material would be in a plasma state, consisting of with positive ions and negative electrons. The goal for all fusioneers is to cause those ions to collide and fuse in sufficient numbers to produce useful energy. This flow of plasma will have about 150 million ℃ \cite{ongena2016magnetic} and should be kept away from damaging the surface of controllers. In this context of controlling, magnetic confinement fusion is proposed. Typical magnetic confinement fusion devices include tokamaks and stellarators. Such devices use strong magnetic fields to keep this plasma sufficiently far away from the wall, so called 'magnetic confinement' \cite{ongena2016magnetic}. Rather than trying to confine the plasma by a magnetic field, inertial confinement fusion begins with a very cold pellet of solid deuterium and tritium and blasts it with high-energy pulsed lasers or particle beams that implode it, suddenly creating an extremely hot, dense plasma in which fusion events occur rapidly \cite{moynihan2023fusion}. Specific examples of this technique include laser-driven inertial confinement fusion, which uses high-energy laser beams; particle beam-driven inertial confinement fusion, employing intense streams of particles; and heavy ion fusion, where heavy ions are accelerated and directed at the target. Magnetized target fusion is an approach that is intermediate between these two methods above \cite{moynihan2023fusion, kirkpatrick1997magnetized}. The plasma would be neither cold and compressed as in ICF, nor as hot as in a tokamak. However, both strong magnetic fields in MCF and some form of implosion like ICF are needed. Hence, it is a 'mixed confinement'. Plasma jet magneto-inertial fusion is a typical example of magnetized target fusion. 





To be specific, controlled fusion devices include pinches, magnetic mirrors, cusps, tokamaks and stellarators, plasmoids, inertial confinement, plasma jet magneto inertial fusion, inertial electrostatic confinement \cite{moynihan2023fusion}. Starting by pinch machines, a kind of early magnetic confinement fusion, such as the Z-pinch \cite{shumlak2020z}, theta pinch, and screw pinch, are early forms of magnetic confinement fusion devices. These machines use strong magnetic fields to compress or "pinch" a plasma, thereby heating it to the point where fusion reactions can occur \cite{ moynihan2023fusion}. Magnetic mirror is a kind of technique to create some special magnetic field which can also be used on magnetic confinement fusion \cite{post1987magnetic}. The magnetic field created by magnetic mirror is strong in surrounding and weak in center. Such characteristic can be to trap the plasma in magnetic confinement fusion.  Similarly to magnetic mirror technique, cusp system is using magnetic cusps, kind of source of magnetic field, face by face to form a magnetic field that strong on either side but tame in the middle, which is also a Magnetic confinement device \cite{mahadevia2014bicuspid, moynihan2023fusion, haines1977plasma}. Tokamaks are famous MCF devices. It has donut-shaped. But also, there are some tokamaks with apple-shaped called sphere tokamaks. In tokamaks, a toroidal magnetic fields are used to accelerate and heat the core plasma. The motion of plasma will generate the poloidal magnetic field which confine the plasma themselves, as shown in \ref{fig:TokamakDonut}. Plasmoid is also a kind of MCF that use some plasma structure on controlled fusion \cite{moynihan2023fusion}. The main idea of ICF is using some high-energy pulsed implosion to make the fusion material congests together to create a fusion environment for a moment. The driven implosions vary, like lasers or particle beams. The method can cause fusion with cold pellet. It is thought to be promising but yields and performances are much worse than tokamaks in practice \cite{cerniauskaite2011systematic}. Plasma Jets and Magneto-Inertial Fusion PJMIF is a typical MTF also thought to be promising. As discussed above, PJMIF is an approach that is dealing intermediate plasma between MCF and ICF. The plasma would be neither cold and compressed as in ICF, nor as hot as in a tokamak. 

Among all those devices, our report focuses on magnetic confinement fusion, especially on tokamak devices, because tokamaks represent the most promising controlled fusion device. A fusion plasma temperature of about 100 million degrees Celsius is required for an efficient fusion reaction rate \cite{li2021experimental}. Among all the fusion devices recently in the world, tokamaks' performance are the closest to this threshold. They have demonstrated sustained plasma confinement times and temperatures necessary for achieving net positive energy output, which is critical for practical energy production. Their relatively well-understood physics and engineering principles, along with ongoing advancements in tokamak design and technology, make them the leading candidates for future fusion power plants.
\section{Computational Modelling on Tokamaks}
\label{section1.3}
Tokamaks represent the most promising controlled fusion concept in the world by now. The strongest tokamak performance so far is the EAST. It can sustainably run for 17 minutes at 70 million ℃ in 2022 \cite{moynihan2023fusion,gong2024overview} or hold on to 126 million ℃ for more than 100s in 2021 \cite{CAS2021}. The currently constructing International Thermonuclear Experimental Reactor (ITER) in the south of France under an international collaboration will become the largest tokamak device in the world \cite{moynihan2023fusion}. Building such an impressive tokamak device will be difficult. Computational modelling will help in designing tokamak reactors. Computational modelling benefits tokamak designs from modelling the turbulent transport to understand the mass \cite{zanisi2024efficient}, proposing a better confinement situation \cite{ding2024high}, analyzing the disrupt causes by statistic modelling \cite{de2011survey}, improving overall design and architectural layout through simulations \cite{federici2016overview}. The plasma in tokamaks is made of positive ions and negative electrons. The plasma in tokamaks is formed because of extreme high temperature. Such that, ionization happens and electron are separated from their nuclei. Only with such high temperature at least 1 million Celsius \cite{li2021experimental}, nuclei will have enough molecular kinetic energy to fusion. In the point of view of plasma, the methods of simulation include one particle models which analyze the action of tokamak devices on a single particle, fluid dynamics models which treat the plasma as a normal fluid interacting with charged particles, ideal magnetohydrodynamics models which regard the plasma as ideal plasma and is entirely dictated by the applied magnetic field, two fluids models which regard the plasma with two fluids,  made by positive ions and negative electrons \cite{ongena2016magnetic,moynihan2023fusion}. Other kinds of methods for modeling tokamaks like kinetic models which tracking the distribution functions of individual particles in phase space to learn the particles interaction, transport models which focus on macroscopic heat or momentum transportation, multi-physics which use combined models of different kinds of materials and integrated models just use combined models. 

Some equipment like the limiters and divertors increase the complexity in modelling tokamaks. Limiters and divertors are used to protect the wall from erosion by plasma particles, shown in Figure \ref{fig:LimiterDivertor}. They are used to filter those particles close to wall. However, the capturing of limiter and another magnetic field in divertor increase the complexity computational modeling. 
\section{Project Overview}
Our project focuses on magnetohydrodynamic (MHD) simulations in complex geometries within tokamak devices, specifically considering interactions with the walls. In this project, we employ an ideal MHD model in two dimensions. We assume that the plasma in tokamaks is ideal, meaning it exhibits no resistivity or viscosity. Perturbations within the plasma are negligible compared to the external magnetic field, simplifying the mathematical representation of the MHD model \cite{moynihan2023fusion}.

This ideal MHD model is particularly valid from a macroscopic perspective. It is most accurate in the core region of the tokamak, where the temperature and density are high. However, it is less valid near the edge of the tokamak, where the plasma interacts with the tokamak walls and divertors. Despite these limitations, we utilize the ideal MHD model in this project due to its simplicity and ease of implementation. Moreover, it provides valuable insights that can guide more detailed future studies.

Beyond tokamaks, ideal MHD is also applied in the analysis of stellar behavior in astrophysics. Interestingly, there is a close relationship between tokamaks and stars, as some tokamaks are modifications of stellarators initially designed for astrophysical experiments. Additionally, ideal MHD is used in research on lightning, semiconductor manufacturing, and the high atmosphere in environmental science.

This is the first report of the project. In this report, we are going to discuss some background theories in chapter \ref{chapter 2} and how numerical techniques are applied based on these theories in chapter 3. Some tests is used to validate these techniques in chapter 4. At last, there is a short summary and some expectation for the next report in chapter 5.

\nomenclature[z-MHD]{MHD}{Magnetohydrodynamics}
\nomenclature[z-MCF]{MCF}{Magnetic Confinement Fusion}
\nomenclature[z-ICF]{ICF}{Inertial Confinement Fusion}
\nomenclature[z-MTF]{MTF}{Magnetized Target Fusion}
\nomenclature[z-MHD-HLLC]{MHD-HLLC}{A MHD solver proposed by \cite{li2005hllc}}