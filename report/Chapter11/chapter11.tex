%!TEX root = ../thesis.tex
%*******************************************************************************
%*********************************** First Chapter *****************************
%*******************************************************************************

\chapter{Summary}  %Title of the First Chapter

\ifpdf
    \graphicspath{{Chapter11/Figs/Raster/}{Chapter11/Figs/PDF/}{Chapter11/Figs/}}
\else
    \graphicspath{{Chapter11/Figs/Vector/}{Chapter11/Figs/}}
\fi

\label{chapter 11}


%************************************************************************** 

  The second report extends the work from the first report by focusing on the resistive wall conditions within tokamaks. The report begins with a review of the ideal Magnetohydrodynamics (MHD) model and highlights the transition from perfect conducting wall conditions to resistive wall conditions. Subsequently, the numerical methods and simulations are adapted to account for the resistivity of materials used in tokamak walls. These methods are validated through a series of computational tests, with comparisons drawn against results obtained under the assumption of a perfectly conducting wall, and are subsequently applied in a simulation within a tokamak-shaped vessel. The report concludes with a discussion on the limitations of electromagnetic dynamics. These findings may provide a more accurate boundary condition for resistive walls. Future studies could potentially propose a more consistent boundary condition that applies across insulator walls, resistive walls, small resistivity walls and perfect conducting walls.