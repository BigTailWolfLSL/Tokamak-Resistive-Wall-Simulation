% ************************** Thesis Abstract *****************************
% Use `abstract' as an option in the document class to print only the titlepage and the abstract.
\begin{abstract}
This thesis presents a detailed exploration of the theoretical foundations and numerical techniques necessary for simulating ideal magnetohydrodynamic (MHD) phenomena in tokamak devices. The primary objective is to introduce the methodologies employed in these simulations and validate their effectiveness through a series of computational experiments. The core numerical schemes implemented include the MHD-HLLC solver and the MUSCL-Hancock method, with a mixed hyperbolic/parabolic GLM divergence cleaning method to ensure a divergence-free magnetic field. Rigid bodies are established with level set method and some boundary conditions like reflective boundary condition and perfect conducting condition are applied on rigid bodies. Validation is achieved through several benchmark tests, including the Orszag-Tang test, shock diffraction tests over wedges and cylinders, and rotated Sod and Brio-Wu tests. These tests confirm the robustness and reliability of the implemented numerical methods. While this study successfully introduces and validates the theoretical and numerical foundations for ideal MHD simulations in rigid body geometries under perfect conducting wall conditions, future work will focus on extending these results to more complex scenarios.

The report 2 extends the foundational work on ideal Magnetohydrodynamics (MHD) by incorporating the effects of resistive wall conditions within tokamak devices. The study begins with a review of the transition from perfect conducting walls to resistive walls, highlighting the implications for plasma stability and magnetic field behavior. Through the application of advanced numerical methods, the simulations account for the resistivity of materials used in tokamak walls, providing a more realistic model of these complex systems. The approaches are validated through computational tests, which compare the results with scenarios that assume perfect conducting conditions. The findings suggest that incorporating resistive walls leads to more accurate boundary conditions and offers new insights for future research on optimizing tokamak performance and stability under varied conditions.


\end{abstract}
